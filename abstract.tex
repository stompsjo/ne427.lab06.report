\begin{abstract}

Significant efforts are taken to protect against radiation damage is many different modes and occupations. From nuclear reactor workers to medical professionals shielding protects against harms from radiation while utilizing the power of various nuclei and their properties. In order to determine economically and practical shielding requirements, studies of $\alpha$ and $\gamma$ radiation deposition is required. This work attempts to quantify how deep $\alpha$ particle radiation from ${}^{241}Am$ and $\gamma$ radiation from ${}^{198}Au$ can penetrate. This can be accomplished by calculating an $\alpha$ particle range and a mass attenuation coefficient for two common $\gamma$ shielding materials: lead and aluminum. Using an enclosed chamber that allows for pressure modulation, a range can be linearly extrapolated. A NaI scintillation detector can be used to record the magnitude of $\gamma$ radiation while varying several thicknesses of shielding material. The effectiveness at blocking this radiation as a function of thickness can result in an extrapolated mass attenuation coefficient. The measured range was $R_{\alpha}$ = 4.81 $\pm$ 0.43 cm. For lead, the mass attenuation coefficient was measured to be $(\frac{\mu}{\rho})$ = 1.98 $\pm$ 0.04 $\times 10^{-4}$ $(\frac{mg}{cm^2})^{-1}$ and for aluminum $(\frac{\mu}{\rho})$ = 8.72 $\pm$ 0.24 $\times 10^{-5}$ $(\frac{mg}{cm^2})^{-1}$. These results are generally in agreement with expected values measured elsewhere. As a result, shielding performance has been quantified in a way that allows engineers to design safe and effective technology.

\end{abstract}
